During the last few years, the efficiency of solar cells could be raised significantly through research and development on different aspects of the solar cell. Besides reducing electrical losses and improving the materials used in solar cells (such as the material for the semi-conductor), we can also try to improve optical properties of the solar cell. On the one hand-side, it is important to reduce reflections and on the other hand-side, it is important to guide as much light as possible to the optical active areas. 
One issue concerning this is cloaking the contact fingers and bus-bars on solar cells to guide the light, which would have hit the contact grid on solar cells, to the optical active areas. This can be used to enhance the efficiency of solar cells. 

With my Bachelorthesis, I want to investigate in the two dimensional design of a surface to get the contact-fingers and the bus-bars cloaked. It is important to cover  a as large as possible angular-acceptance as well al a as homogeneous as possible light distribution on the active area of the solar cell. The design we want to concentrate on is a Polymer structure (n=1.5), which should have the right refraction-properties. In this Bachalorthesis, we will not take reflection into account. 
First, I want to design a continuous solution in two dimensions based on a given one dimensional solution. \cite{schumann2015cloaked}
Second, I want to try to find an optical design based on Fresnel-optics to cloak the contact-fingers and the busbars. 
In the end, I want to design a meta-material to cloak the contact fingers and the busbars and compare it with the first two designs.. 
